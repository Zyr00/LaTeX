\documentclass{article}

\usepackage{fancyhdr}
\usepackage{amsmath}

\pagestyle{fancy}
\fancyhf{} % apagar as configurações
\rhead{Tunes}
\rfoot{Página \bfseries\thepage}

\newcommand{\bi}[1]{\textbf{\textit{#1}}}

\newenvironment{realce}{$\Rightarrow$}{$\Leftarrow$}

\begin{document}

  \section{normal stuff}

  Texto normal com codificação \emph{latin1} \\
  \texttt{monotype} \\
  \textbf{bold} \\
  \textit{italic} \\
  \textsc{SmallCaps} \\
  \emph{emphase} \\

  \begin{center}
    Texto centrado\marginpar{notas margin}
  \end{center}

  \begin{flushright}
    Texto à direita\footnote{footnote}
  \end{flushright}

  \begin{itemize}
    \item item 1
    \item item 2
    \item item 3
  \end{itemize}

  \begin{enumerate}
    \item item 1
    \item item 2
    \item item 3
  \end{enumerate}

  \begin{description}
    \item[foo] item 1
    \item[bar] item 2
    \item[zbr] item 3
  \end{description}

  Número bi: \bi{123123123}

  \clearpage{}

  \section{%
    \begin{realce}
      Formulas:
    \end{realce}
  }

  \paragraph{}
  As formulas podem ser in-line com o assim $ {(a + b)}^2 = a^2 + 2ab + b^2 $ ou assim
  \[ {(a + b)}^2 = a^2 + 2ab + b^2 \]

  \paragraph{}
  \begin{equation}
    \centering
    \label{Firs}
    a_0 + \cdots + a_n + a_{n + 1}
  \end{equation}

  \begin{equation}
    \centering
    \sqrt{\frac{1}{2}}
  \end{equation}

  \begin{equation}
    \centering
    \lim_{n \to \infty}\sum_{k = 1}^n
    \frac{1}{k^2} = \frac{\pi^2}{6}
  \end{equation}

  \begin{equation}
    \centering
    \forall x \in \mathbf{R} \qquad x^{2} \geq 0
  \end{equation}

  \begin{equation}
    \centering
    v = \sigma_1 \cdot\sigma_2 \tau_1 \cdot\tau_2
  \end{equation}

  \begin{equation}
    \centering
    \lim_{x \rightarrow 0} \frac{\sin x}{x} = 1
  \end{equation}

  \begin{equation}
    \centering
    1 + {(\frac{1}{1-x^{2}})}^3
  \end{equation}

  \begin{equation}
    \centering
    1 + {\left( \frac{1}{1-x^{2}} \right)}^3
  \end{equation}

  \begin{equation}
    \centering
    \mathbf{x} = {\left(%
        \begin{array}{ccc}
        x_{11} & x_{12} & \ldots \\
        x_{21} & x_{22} & \ldots \\
        \vdots & \vdots & \ddots
        \end{array}
        \right)}
  \end{equation}

  \begin{equation}
    \centering
    1 - \frac{100^{a \times b}}{\frac{\alpha + \beta}{\Omega}}
  \end{equation}

  \begin{equation}
    \centering
    \mid x \mid =
    \begin{cases}
      x  & (x \geq 0) \\
      -x & (x < 0)
    \end{cases}
  \end{equation}

\end{document}
