\documentclass[11pt, arial]{article}

\usepackage{graphicx}
\usepackage{listings}
\usepackage{xcolor}

\definecolor{codegreen}{rgb}{0,0.6,0}
\definecolor{codegray}{rgb}{0.5,0.5,0.5}
\definecolor{codepurple}{rgb}{0.58,0,0.82}

\lstdefinestyle{mystyle}{%
commentstyle=\color{codegray},
keywordstyle=\color{blue},
stringstyle=\color{codepurple},
basicstyle=\ttfamily\small,
breakatwhitespace=false,
breaklines=true,
captionpos=b,
keepspaces=true,
numbers=left,
% numbersep=5pt,
showspaces=false,
showstringspaces=true,
showtabs=true,
tabsize=2,
frame=single,
rulecolor=\color{black}
}

\lstset{style=mystyle, literate=
  {á}{{\'a}}1 {é}{{\'e}}1 {í}{{\'\i}}1 {ó}{{\'o}}1 {ú}{{\'u}}1
  {Á}{{\'A}}1 {É}{{\'E}}1 {Í}{{\'I}}1 {Ó}{{\'O}}1 {Ú}{{\'U}}1
  {à}{{\`a}}1 {è}{{\`e}}1 {ì}{{\`\i}}1 {ò}{{\`o}}1 {ù}{{\`u}}1
  {À}{{\`A}}1 {È}{{\'E}}1 {Ì}{{\`I}}1 {Ò}{{\`O}}1 {Ù}{{\`U}}1
  {ä}{{\"a}}1 {ë}{{\"e}}1 {ï}{{\"\i}}1 {ö}{{\"o}}1 {ü}{{\"u}}1
  {Ä}{{\"A}}1 {Ë}{{\"E}}1 {Ï}{{\"I}}1 {Ö}{{\"O}}1 {Ü}{{\"U}}1
  {â}{{\^a}}1 {ê}{{\^e}}1 {î}{{\^\i}}1 {ô}{{\^o}}1 {û}{{\^u}}1
  {Â}{{\^A}}1 {Ê}{{\^E}}1 {Î}{{\^I}}1 {Ô}{{\^O}}1 {Û}{{\^U}}1
  {Ã}{{\~A}}1 {ã}{{\~a}}1 {Õ}{{\~O}}1 {õ}{{\~o}}1
  {œ}{{\oe}}1 {Œ}{{\OE}}1 {æ}{{\ae}}1 {Æ}{{\AE}}1 {ß}{{\ss}}1
  {ű}{{\H{u}}}1 {Ű}{{\H{U}}}1 {ő}{{\H{o}}}1 {Ő}{{\H{O}}}1
  {ç}{{\c c}}1 {Ç}{{\c C}}1 {ø}{{\o}}1 {å}{{\r a}}1 {Å}{{\r A}}1
  {€}{{\euro}}1 {£}{{\pounds}}1 {«}{{\guillemotleft}}1
  {»}{{\guillemotright}}1 {ñ}{{\~n}}1 {Ñ}{{\~N}}1 {¿}{{?`}}1
}

\title{%
  Guião N.º 2 \\[0.5cm]
  \textbf{\textit{Threads} em Java}
}
\author{João Cunha \\8170348@estg.ipp.pt \\\includegraphics[width=5cm]{scrot}}

\begin{document}
  \maketitle
  \section{\textit{Threads}}
  Uma \textit{thread} pode ser vista como uma linha de execução de um processo.
  Um processo é algo que dispõem de um segmento de memória próprio e que executa
  de forma independente dos restantes processos. Um processo contém pelo menus uma
  linha de execução (ou thread). \textit{Threads} são frequentemente consideradas como
  \textit{lightweight processes} já que tanto os processos como as \textit{threads}
  possibilitam a execução de código. Contudo, uma \textit{Thread} existe dentro do contexto
  de um processo e partilha o segmento de memória, ficheiros abertos e demais recursos, com as
  restantes \textit{threads} (caso existam) do mesmo processo.

  \lstinputlisting[language=Java, caption=Implementação do interface \textit{Runnable}]{HelloRunnable.java}
\end{document}
